\documentclass[xcolor=dvipsnames, aspectratio=169, 14pt]{beamer}
\usefonttheme{professionalfonts}

\usepackage{xltxtra} % lädt auch fixltx2e, etex, fontspec, xunicode …

\usepackage{array, color, enumerate, fixmath, gensymb, graphicx, graphics, icomma, mathcomp, mathtools, minted, multirow, nicefrac, placeins, relsize, rotating, tabularx, tikz, units, xspace, xtab, lipsum}

\usepackage[math-style=upright]{unicode-math}

% Schriften
\setmathfont{Neo Euler}
\setromanfont[Mapping=tex-text]{Aller}
\setsansfont[Mapping=tex-text]{Aller}
\setmonofont[Scale=0.96]{Iosevka}
\newfontfamily\titlefamily{Aller Display}

% Statt babel:
\usepackage{polyglossia}
\setdefaultlanguage[variant = uk]{english}

% sonstige Einstellungen
\usetheme[height=14mm]{Rochester}
\useinnertheme{circles}
\setbeamertemplate{navigation symbols}{}

\setbeamerfont{block title}{family=\titlefamily}
\setbeamertemplate{blocks}[rounded]

\setbeamerfont{frametitle}{family=\titlefamily}

%Presentation
\usecolortheme[accent=red]{solarized}
\setbeamercolor*{frametitle}{fg = solarized@red, bg = solarized@rebase02}
\setbeamercolor{normal text}{fg = solarized@rebase3, bg = solarized@rebase03}

\definecolor{pagenumbers}{HTML}{B4EBE7}

\addtobeamertemplate{headline}{}
{%
	\begin{tikzpicture}[remember picture,overlay]
		\node[anchor=south east] at (current page.south east) {\textcolor{pagenumbers}{\insertframenumber}};
	\end{tikzpicture}%
}

\renewcommand{\tabularxcolumn}[1]{>{\centering\arraybackslash}m{#1}}

\usemintedstyle{solarizedlight}
\newminted{python3}{
% 		fontsize = \scriptsize, 
% 		linenos, stepnumber = 5,
		frame = lines, framesep = 8pt,
		baselinestretch = 1.2,
		numbersep = 8pt,
		gobble = 1,
		bgcolor = solarized@rebase02,
		obeytabs, tabsize = 4,
		stripnl = false
		}
% \includeonlyframes{current} %[label=current]

\begin{document}

\begin{frame}[plain]
\begin{center}

\vfill

\textlarger[4]
{\color{solarized@red}
	\titlefamily Generatorfunktionen\strut{}\\
	und das Yield-Statement
}

\vfill

Gerrit Ansmann

\vfill
\end{center}
\end{frame}

\begin{frame}[fragile]{Syntax (Generator Comprehensions)}
	\begin{columns}%
	\column{0.5\linewidth}
		Liste:
	\begin{python3code}
	X = [n**2 for n in range(4)]
	
	for element in X:
		print(element)
	# 0
	# 1
	# 4
	# 9
	\end{python3code}
	\column<2->{0.5\linewidth}
		Generator:
	\begin{python3code}
	X = (n**2 for n in range(4))
	
	for element in X:
		print(element)
	# 0
	# 1
	# 4
	# 9
	\end{python3code}
	\end{columns}
\end{frame}

\begin{frame}<1-2>[label=comparison]{Listen vs. Generatoren – Eigenschaften}
	\begin{columns}%
	\column{0.5\linewidth}
		Listen:
		\vspace{0.5\baselineskip}
		\begin{itemize}[topsep=0.5\baselineskip]
			\setlength{\itemsep}{0.5\baselineskip}
			\item<alert@2> alle Elemente werden gespeichert
			\item<alert@3> Elemente direkt ansprechbar
			\item<alert@4> können beliebig oft\\ genutzt werden
			\item<alert@5> haben eine Länge (\texttt{len})
		\end{itemize}
	\column{0.5\linewidth}
		Generatoren:
		\begin{itemize}
		\vspace{0.5\baselineskip}
			\setlength{\itemsep}{0.5\baselineskip}
			\item<alert@2> erzeugen Elemente bei Bedarf
			\item<alert@3> Elemente nicht direkt ansprechbar
			\item<alert@4> können nur einmal durchlaufen werden
			\item<alert@5> haben keine explizite Länge
		\end{itemize}
	\end{columns}
\end{frame}


\begin{frame}[fragile]{Speichern}
	\vspace{-\baselineskip}
	\begin{columns}%
	\column{0.5\linewidth}
	\begin{python3code}
	X = [n for n in range(10**9)]
	
	print(1 in X)
	# dauert und frisst Speicher
	\end{python3code}
	
	\column{0.5\linewidth}
	\begin{python3code}
	X = (n for n in range(10**9))
	
	print(1 in X)
	# flutscht
	\end{python3code}
	\end{columns}
	
	\vspace{-\baselineskip}
	
	\begin{columns}<2->%
	\column{0.5\linewidth}
	\begin{python3code}
	X = [n for n in range(10**9)]
	
	print(10**10 in X)
	# dauert und frisst Speicher
	\end{python3code}
	
	\column{0.5\linewidth}
	\begin{python3code}
	X = (n for n in range(10**9))
	
	print(10**10 in X)
	# dauert
	\end{python3code}
	\end{columns}
\end{frame}

\begin{frame}[fragile]{Generierung nach Bedarf}
	\begin{columns}%
	\column[t]{0.5\linewidth}
	\begin{python3code}
	X = [1/n for n in range(9)]
	# ZeroDivisionError
	\end{python3code}
	
	\column[t]{0.5\linewidth}
	\begin{python3code}
	X = (1/n for n in range(9))
	
	next(X)
	# ZeroDivisionError
	\end{python3code}
	\end{columns}
\end{frame}

\againframe<3>{comparison}

\begin{frame}[fragile]{Elemente Ansprechen}
	\begin{columns}%
	\column{0.5\linewidth}
	\begin{python3code}
	X = [n**2 for n in range(9)]
	
	print(X[4])
	# 16
	\end{python3code}
	
	\column{0.5\linewidth}
	\begin{python3code}
	X = (n**2 for n in range(9))
	
	print(X[4])
	# TypeError
	\end{python3code}
	\end{columns}
\end{frame}

\againframe<4>{comparison}

\begin{frame}[fragile]{Wiederbenutzbarkeit}
	\begin{columns}%
	\column{0.5\linewidth}
	\begin{python3code}
	X = [n**2 for n in range(9)]
	
	print(sum(X))
	# 204
	
	print(sum(X))
	# 204
	\end{python3code}
	
	\column{0.5\linewidth}
	\begin{python3code}
	X = (n**2 for n in range(9))
	
	print(sum(X))
	# 204
	
	print(sum(X))
	# 0
	\end{python3code}
	\end{columns}
\end{frame}

\againframe<5>{comparison}
\begin{frame}[fragile]{Länge}
	\begin{columns}%
	\column[t]{0.5\linewidth}
	\vspace{-\baselineskip}
	\begin{python3code}
	X = [n**2 for n in range(9)]
	
	print(len(X))
	# 9
	\end{python3code}
	
	\column[t]{0.5\linewidth}
	\vspace{-\baselineskip}
	\begin{python3code}
	X = (n**2 for n in range(9))

	länge = 0
	for entry in X:
		länge += 1
		foo(X)

	print(länge)
	# 9
	\end{python3code}
	\end{columns}
\end{frame}

\begin{frame}{Die Qual der Wahl}
	Generatoren bieten sich an, wenn:
	\vfill
	\begin{itemize}
		\setlength{\itemsep}{\fill}
		\item alle Elemente zusammen viel Speicher benötigen
		\item nicht alle Elemente benötigt werden\\ (oder nicht sicher ist, ob die Iterable überhaupt benötigt wird)
		\item die Iterable nur einmal durchlaufen wird
		\item für viele Operationen mit Iterablen
		\item nichts gegen sie spricht (weil pythonischer)
	\end{itemize}
	\vfill
	In Python~3 geben viele Standardfunktionen Generatoren aus.
\end{frame}

\begin{frame}{Die Qual der Wahl}
	Generatoren bieten sich nicht an, wenn:
	\vfill
	\begin{itemize}
		\setlength{\itemsep}{\fill}
		\item die Iterable mehrfach benötigt wird\\ \textbf{und} die Elemente aufwendig zu generieren sind
		\item die Iterable verändert werden soll
		\item die Elemente in anderer Reihenfolge benötigt werden
		\item als Ersatz für NumPy-Arrays,\\ wenn die Effizienz der Operationen relevant ist
	\end{itemize}
\end{frame}

\begin{frame}[fragile]{Generatoren manuell Durchlaufen}
	\begin{columns}%
	\column{0.5\linewidth}
	\vspace{-\baselineskip}
	\begin{python3code}
	X = irgendeine_liste
	
	index = 0
	while foo():
		if bar():
			if index<len(X):
				print(X[index])
				index += 1
			else:
				break
	\end{python3code}
	
	\column{0.5\linewidth}
	\vspace{-\baselineskip}
	\begin{python3code}
	X = irgendein_generator
	
	while foo():
		if bar():
			try:
				print(next(X))
			except StopIteration:
				break
	\end{python3code}
	\end{columns}
\end{frame}

\begin{frame}[fragile]{Verketten von Generatoren}
	\begin{columns}%
	\column{0.5\linewidth}
	\vspace{-\baselineskip}
	\begin{python3code}
	X,Y,Z = irgendwelche_listen
	
	gesamt = X+Y+Z
	# legt neues Objekt an
	\end{python3code}
	
	\column{0.5\linewidth}
	\vspace{-\baselineskip}
	\begin{python3code}
	X,Y,Z = irgndwelche_genratorn
	
	from itertools import chain
	gesamt = chain(X,Y,Z)
	\end{python3code}
	\end{columns}
\end{frame}

\begin{frame}[fragile]{Generatoren selbstgemacht}
	\begin{columns}%
	\column{0.5\linewidth}
	\vspace{-\baselineskip}
	\begin{python3code}
	def my_range(länge):
		ergebnis = []
		i = 0
		while i < länge:
			ergebnis.append(i)
			i += 1
		return ergebnis
	\end{python3code}
	
	\column{0.5\linewidth}
	\vspace{-\baselineskip}
	\begin{python3code}
	def my_range(länge):
		
		i = 0
		while i < länge:
			yield i
			i += 1
		
	\end{python3code}
	\end{columns}
\end{frame}

\begin{frame}[fragile]{Generatoren selbstgemacht}
	\begin{columns}%
	\column{0.5\linewidth}
	\vspace{-\baselineskip}
	\begin{python3code}
	def nochngenerator():
		print("null")
		yield None
		print("eins")
		yield 2
		print("zwei")
		for n in range(3,10):
			yield n
			print(n)
	\end{python3code}
	
	\column{0.5\linewidth}
	\vspace{-\baselineskip}
	\begin{python3code}
	X = nochngenerator()
	next(X)
	# null
	next(X)
	# eins
	next(X)
	# zwei
	next(X)
	# 3
	\end{python3code}
	\end{columns}
\end{frame}

\begin{frame}[fragile]{Verschachtelung von Generatoren}
	\begin{columns}%
	\column[t]{0.5\linewidth}
	\vspace{-\baselineskip}
	\begin{python3code}
	X,Y,Z = irgndwelche_genratorn
	
	def alles_nacheinander():
		for x in X:
			yield x
		for y in Y:
			yield y
		for z in Z:
			yield z
	\end{python3code}
	
	\column[t]{0.5\linewidth}
	\vspace{-\baselineskip}
	\begin{python3code}
	X,Y,Z = irgndwelche_genratorn
	
	def alles_nacheinander():
		yield from X
		yield from Y
		yield from Z
	\end{python3code}
	
	(seit Python 3.3)
	\end{columns}
\end{frame}

\begin{frame}[fragile]{Exkurs: Koroutinen}
	\begin{columns}%
	\column{0.5\linewidth}
	\vspace{-\baselineskip}
	\begin{python3code}
	def krümelmonster():
		sättigung = 0
		while sättigung <= 2:
			speise = (yield)
			if "keks" in speise:
				sättigung += 1
		print("Ich bin satt.")
	\end{python3code}
	
	\column{0.5\linewidth}
	\vspace{-\baselineskip}
	\begin{python3code}
	monster = krümelmonster()
	next(monster)
	monster.send("Schokokeks")
	monster.send("Heringspizza")
	monster.send("Sardellenkeks")
	# Ich bin satt.
	# StopIteration
	\end{python3code}
	\end{columns}
\end{frame}

\end{document}

TODO:
2 vs 3
Implementation von range
Comparison with Internet
Iterator vs. Iterable
