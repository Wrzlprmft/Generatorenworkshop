\documentclass[a4paper, 14 pt]{extarticle}
 
\usepackage{amsmath, amssymb}

\usepackage{xltxtra}

\usepackage{array, color, enumerate, fixmath, gensymb, graphicx, graphics, icomma, mathcomp, mathtools, minted, multirow, nicefrac, placeins, relsize, rotating, tabularx, titlesec, units, xspace, xtab, lipsum}

\usepackage[math-style=upright]{unicode-math}



% ----Schriften----
\setmathfont{Neo Euler}
\setromanfont[Mapping=tex-text]{Ubuntu}
\setsansfont[Mapping=tex-text]{Ubuntu}
\setmonofont[Scale=1.13]{Ubuntu Mono}
\newfontfamily\titlefamily{Aller Display}

% ----Sprache----
\usepackage{polyglossia}
\setdefaultlanguage[spelling=new, latesthyphen=true]{german}


% ----Seite einstellen----
\setlength{\columnsep}{1 cm}
\setlength{\textwidth}{17 cm}
\setlength{\evensidemargin}{2 cm} %Abstand Text – äußerer Rand

\setlength{\hoffset}{-1 in}
\setlength{\oddsidemargin}{21 cm}
\addtolength{\oddsidemargin}{-\textwidth}
\addtolength{\oddsidemargin}{-\evensidemargin}

\setlength{\voffset}{-1 in}
\setlength{\topmargin}{1.6 cm}

\setlength{\headsep}{0 pt}
\setlength{\headheight}{0 pt}
\setlength{\textheight}{29.7 cm}
\addtolength{\textheight}{-2 \topmargin}
\addtolength{\textheight}{- \headheight}
\addtolength{\textheight}{- \headsep}
%\addtolength{\textheight}{- \footskip}

\hyphenpenalty=500
\pretolerance=150
\tolerance=1500
\setlength{\emergencystretch}{\textwidth}

% \setlength{\parindent}{0pt} 

\titleformat*{\section}{\titlefamily\larger[2]}
\titleformat*{\subsection}{\larger[1]}

\setcounter{secnumdepth}{0}

\thispagestyle{empty}

\usemintedstyle{bw}
\newminted{python3}{
% 		fontsize = \scriptsize, 
% 		linenos, stepnumber = 5,
		frame = lines,
		framerule = 1pt,
		framesep = 8pt,
		baselinestretch = 1.2,
		numbersep = 8pt,
		gobble = 0,
		obeytabs, tabsize = 4}


\begin{document}
\section{Übungen}
\vfill
\subsection{Wiederholung zum Warmwerden}
Vergewissere Dich kurz, dass:
\begin{itemize}
	\item Generatoren Elemente erst bei Bedarf erzeugen.
	\item Generatoren nur einmal genutzt werden können.
	\item Generatoren kaum Speicher verbrauchen.
\end{itemize}

\vfill

\subsection{Spaß mit \texttt{next}, \texttt{StopIteration} und \texttt{itertools.chain}}
Schreibe eine Funktion, die einen Generator als Argument nimmt und:
\begin{itemize}
	\item falls der Generator nur ein einziges Element generiert:\\
		das Element zurückgibt;
	\item falls der Generator mehr als ein Element generiert:\\
		eine zum Generator äquivalente Iterable zurückgibt\\ (ohne den Generator komplett zu durchlaufen).
\end{itemize}

\vfill

\subsection{Generatorfunktionen und \texttt{yield}}
Ersetze folgende Funktion, die eine Liste zurückgibt,\\ durch eine analoge Generatorfunktion.
\begin{python3code}
	def listenbastler():
		liste = [42]
		
		for i in range(666):
			liste.append(i**2)
		
		liste.extend(range(23))
		liste.append(17)
		
		return liste
\end{python3code}

\subsection{Unendliche Generatoren}
Schreibe eine Generatorfunktion, die untentwegt aufwärtszählt.\\
Schaue Dir dabei nicht die Dokumentation von \texttt{itertools.count} an.

\end{document}
